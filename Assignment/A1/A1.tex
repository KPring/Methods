\documentclass[11pt,a4paper]{article}
\usepackage[a4paper,left=2.5cm,right=2.5cm,bottom=3cm,top=2.5cm]{geometry}
\usepackage[latin1]{inputenc}
\usepackage{setspace,dsfont,amsmath,amssymb}
\usepackage[longnamesfirst,nonamebreak]{natbib}
\usepackage{longtable,booktabs,pdfpages}
\usepackage[english]{babel}
\usepackage{eurosym,multirow,hyperref,placeins}
\usepackage[lastexercise]{exercise}

\hypersetup{
    colorlinks,%
    citecolor=blue,%
    filecolor=black,%
    linkcolor=blue,%
    urlcolor=black}


\setlength{\textwidth}{6.3in}
\setlength{\textheight}{23.0cm}
\setlength{\oddsidemargin}{0in}
\setlength{\abovetopsep}{2ex}
\newcommand{\Lik}{\mathcal{L}}
\newcommand{\lau}{\lambda_u}
\newcommand{\lae}{\lambda_e}
\newcommand{\1}{\mathbb{1}}
\newcommand{\V}{\mathbb{V}}
\newcommand{\E}{\mathbb{E}}
\newcommand{\de}{\delta}
\newcommand{\ow}{\overline{w}}
\newcommand{\uw}{\underline{w}}
\newtheorem{theorem}{Theorem}
\newtheorem{lemma}{Lemma}
\newtheorem{proposition}{Proposition}
\newtheorem{corollary}{Corollary}

\begin{document}
\doublespacing

\section{Class Assignment: Introduction to R}

\subsection{Introduction}

Using this sample code, 

\begin{verbatim}
install.packages("BB")
library(BB)
source("A1.R")
?for
??rpareto
dir()
1+1
2/2
save.image("misc.RDATA")
1:10
30%%4
setwd("/Users/ms486/Dropbox/Papers/Progress")
getwd()
ls()
2/0
log(-1)
sum(1:10)
\end{verbatim}

\begin{Exercise}[title=Introduction]
\begin{enumerate}
\item Create a directory for this class and store your script ``a0.R" 
\item Install the packages, Hmisc, gdata,boot,xtable,MASS,moments,snow,mvtnorm
\item Set your working directory
\item List the content of your directory and the content of your environment
\item Check whether 678 is a multiple of 9
\item Save your environment
\item Find help on the function mean, cut2
\item Find an operation that returns NaN (Not A Number)
\end{enumerate}
\end{Exercise}

\subsection{Objects}

\paragraph{Vectors, Matrix, Arrays}
\begin{verbatim}
vec0  = NULL
vec1  = c(1,2,3,4)
vec2  = 1:4
vec3  = seq(1,4,1)
vec4  = rep(0,4)
sum(vec1)
str(vec1)
prod(vec1)
mat1  = mat.or.vec(2,2)
mat2  = matrix(0,ncol=2,nrow=2,byrow=T)
mat3  = cbind(c(0,0),c(0,0))
mat4  = rbind(c(1,1),c(0,0))
mat5  = matrix(1:20,nrow=5,ncol=4)
mat5[1:2,3:4]
mat5[1,]
arr1  = array(0,c(2,2))
dim(mat4)
dim(vec2)
length(vec2)
length(mat1)
class(mat4)
\end{verbatim}

\begin{Exercise}[title=Object Manipulation]
\begin{enumerate}
\item Print Titanic, and write the code to answer these questions (one function (sum) , one operation)
\begin{enumerate}
\item Total population
\item Total adults
\item Total crew
\item $3^{rd}$ class children
\item $2^{nd}$ class adult female
\item $1^{st}$ class children male
\item Female Crew survivor
\item $1^{st}$ class adult male survivor
\end{enumerate}
\item Using the function \emph{prop.table}, find
\begin{enumerate}
\item The proportion of survivors among first class, male, adult
\item The proportion of survivors among first class, female, adult
\item The proportion of survivors among first class, male, children
\item The proportion of survivors among third class, female, adult
\end{enumerate}
\end{enumerate}
\end{Exercise}

\begin{Exercise}[title=Vectors - Introduction]
\begin{enumerate}
\item Use three different ways, to create the vectors 
\begin{enumerate}
\item $a=1,2,\ldots,50$
\item $b=50,49,\ldots,1$
\item[Hint]: rev
\end{enumerate} 
\item Create the vectors 
\begin{enumerate}
\item $a=10,19,7,10,19,7,\ldots,10,19,7$ with 15 occurrences of 10,19,7
\item $b=1,2,5,6,\ldots,1,2,5,6$ with 8 occurrences of 1,2,5,6
\item[Hint]: rep
\end{enumerate} 
\item Create a vector of the values of $log(x)sin(x)$ at $x=3.1,3.2,\ldots,6$
\item Using the function sample, draw 90 values between (0,100) and calculate the mean. Re-do the same operation allowing for replacement. 
\item Calculate 
\begin{enumerate}
\item $\sum_{a=1}^{20}\sum_{b=1}^{15}\dfrac{exp(\sqrt{a})log(a^5)}{5+cos(a)sin(b)}$
\item $\sum_{a=1}^{20}\sum_{b=1}^{a} \dfrac{exp(\sqrt{a})log(a^5)}{5+exp(ab)cos(a)sin(b)}$
\end{enumerate}
\item Create a vector of the values of $\exp(x) \cos(x)$ at x = 3, 3.1, ...6.
\end{enumerate}
\end{Exercise}

\begin{Exercise}[title=Vectors - Advanced]
\begin{enumerate}
\item Create two vectors $xVec$ and $yVec$ by sampling 1000 values between 0 and 999.
\item Suppose $xVec = (x_1, \ldots, x_n)$ and $yVec = (y_1, \ldots, y_n)$
\begin{enumerate}
\item Create the vector $(y_2-x_1,\ldots, y_n -x_{n-1})$ denoted by $zVec$.
\item Create the vector $(\dfrac{sin(y_1)}{\cos(x_2},\dfrac{sin(y_2)}{\cos(x_3}, \ldots, \dfrac{sin(y_{n-1})}{\cos(x_n})$ denoted by $wVec$.
\item Create a vector $subX$ which consists of the values of $xVec$ which are $\geq$ 200.
\item What are the index positions in $yVec$ of the values which are $\geq$ 600.
\end{enumerate}
\end{enumerate}
\end{Exercise}


\begin{Exercise}[title=Matrix]
\begin{enumerate}
\item Create the matrix $A  = \left| \begin{array}{ccc}
 1 & 1 & 3 \\
5  & 2 & 6 \\
-2 &-1 & -3 \end{array} \right|$
\begin{enumerate}
\item Check that $A^3$=0 (matrix 0).
\item Bind a fourth column as the sum of the first and third column
\item Replace the third row by the sum of the first and second row
\item Calculate the average by row and column. 
\end{enumerate}
\item Consider this system of linear equations: 
\begin{eqnarray}
2x  +   y  +  3z  =  10   \\ 
 x  +   y  +   z  =   6   \\
 x  +  3y  +  2z  =  13 
 \end{eqnarray}
 \item Solve this equation. 
\end{enumerate}
\end{Exercise}

\begin{Exercise}[title=Functions]
 \begin{enumerate}
 \item Write a function \emph{fun1} which takes two arguments (a,n) where (a) is a scalar and n is a positive integer, and returns
 \begin{equation*}
 a+\dfrac{a^2}{2}+\dfrac{a^3}{3} + \ldots + \dfrac{a^n}{n}
 \end{equation*}
 \item Consider the function 
 \begin{equation}
f(x)  = \left\{ \begin{array}{ll}
         x^2+2x+|x| & \mbox{if $x < 0$};\\
         x^2+3 + \log(1+x) & \mbox{if $0 \leq x < 2$};\\
         x^2 + 4x -14 & \mbox{if $x \geq 2$}.\end{array} \right. 
 \end{equation}
Evaluate the function at -3, 0 and 3. 
\end{enumerate}
\end{Exercise}


\begin{Exercise}[title= Indexes]
\begin{enumerate}
\item Sample 36 values between 1 and 20 and name it $v1$ 
\item Use two different ways, to create the subvector of elements that are not in the first row. \emph{Hint: which and subset can not be used.} Check x[a] and x[-a]. 
\item Create a logical element (TRUE or FALSE), v2, which is true if $v1>5$. Can you convert this logical element into a dummy 1 (TRUE) and 0 (FALSE)?
\item Create a matrix m1 $[6\times 6]$ which is filled by row using the vector v1. 
\item Create the following object 
\begin{verbatim}
x = c(rnorm(10),NA,paste("d",1:16),NA,log(rnorm(10)))
 \end{verbatim}
\item Test for the position of missing values, and non-finite values.
Return a subvector free of missing and non-finite values. 
\end{enumerate}
\end{Exercise}


\begin{Exercise}[title=Data Manipulation]
\begin{enumerate}
\item Load the library AER, and the dataset (data("GSOEP9402")) to be named dat. 
\item What type of object is it? Find the number of rows and column? Can you provide the names of the variables?
\item Evaluate and plot the average annual income by year. 
\item Create an array that illustrates simultaneously the income differences (mean) by gender, school and memployment. 
\end{enumerate}
\end{Exercise}

\begin{Exercise}[title=First regression]
\begin{enumerate}
\item Load the dataset (data("CASchools")) to be named data.
\item Using the function lm, run a regression of read on the following variables: district, school, county, grades, students, teachers, calworks, lunch, computer, expenditure, income and english. Store this regression as reg1. 
\item Can you run a similar regression by specifying, \begin{verbatim}
formula = y ~ x. lm(formula)
\end{verbatim} Create reg2, that uses only the 200 first observations.  
\end{enumerate}
\end{Exercise}

\begin{Exercise}[title=Advanced indexing]
\begin{enumerate}
\item Create a vector \emph{lu} of 200 draws from a pareto distribution (1,1). How many values are higher than 10. Replace these values by draws from a logistic distribution (6.5,0.5). 
\item Create a vector \emph{de} of 200 draws from a normal distribution (1,2). Set $de = \log(de)$, and count the number of missing values or negative values. Replace these values by draws from a normal distribution (0,1) truncated at 0. \emph{hint:truncnorm}
\item Create two vectors, \emph{orig} and \emph{dest} as 200 draws from a uniform distribution [0,1]. 
\item Create two matrices, \emph{hist} and \emph{dist} as 200*200 draws from a uniform distribution [0,1]. 
\item Consider this function

\begin{equation}
q_{jl}(w) =  \frac{r+de_j}{r+de_l} w + lu_j log(w) - lu_l (1+log(w)) + \frac{r+de_j}{r+de_l} \sum_{k \neq j} su_{jk} - \sum_{k \neq l} su_{lk} + \frac{r+de_j}{r+de_l} \sum_{k \neq j} se_{jk} - \sum_{k \neq l} se_{lk}
\end{equation}

where 
\begin{eqnarray}
su_{j,l} = \log(orig_j + dest_l + dist_{j,l})/(1+\log(orig_j + dest_l + dist_{j,l}))\\
se_{j,l} = \exp(orig_j + dest_l + hist_{j,l})/(1+\exp(orig_j + dest_l + hist_{j,l}))
\end{eqnarray}
\item Create the matrices su and se.
\item Set r = 0.05. Create a function to evaluate $q_{jl}(.)$. Evaluate $q_{jl}(9245)$ for all pairs (j,l). 
\item Create gridw, which consists of a sequence from 9100 to 55240 of length 50. 
\item Using the function sapply, evaluate $q_{jl}$. Store the ouput into an array of dimension $(50\times 200 \times 200)$. How long does it take to evaluate $q_{jl}()$ for each value of w? 
\end{enumerate}
\end{Exercise}


\paragraph{List}

\begin{verbatim}
li      = list()
li[[1]] = mat1
li[[2]] = Titanic
li1     = list(x=mat1,y=Titanic)
li1$x
li2$y
\end{verbatim}

\paragraph{Dataframe}

\begin{verbatim}
data=data.frame(x=rnorm(100),y=runif(100))
data
browse(data)
edit(data)
data[,1]
data[1,]
data$x
names(data)
attach(data)
x
detach(data)
y
\end{verbatim}

\paragraph{Tests and Conversion}

\begin{verbatim}
is.na()
is.list()     as.list()
is.factor()   as.factor()	  
is.matrix()   
is.vector()
is.array()
is.finite()
a==b
a=>b
a<=b
\end{verbatim}

\begin{Exercise}[title = Tests and indexing]
\begin{enumerate}
\item Test if $c(1,2,3)$ is an array? a vector? a matrix?  
\item x0 = rnorm(1000); Using the function \emph{table()} count the number of occurrences of $x0>0$, $x0>1$, $x0>2$, $x0>0.5$, $x0<1$ and $x0>-1$
\item \begin{verbatim}
x1 = cut2(runif(100,0,1),g=10) 
levels(x1)=paste("q",1:10,sep="")
\end{verbatim}
\item Test whether or not x1 is a factor?
\item Verify that "q1" has 10 occurences.
\item Convert x1 into a numeric variables. What happens to the levels?
\item \begin{verbatim}
rand = rnorm(1000)
\end{verbatim}
\item Using the function \emph{which()} find the indexes of positive values.
\item Create the object w of positive values of x using:
\begin{enumerate}
 \item Which
 \item Subset
 \item By indexing directly the values that respect a condition 
 \end{enumerate} 
\end{enumerate}
\end{Exercise}


\subsection{Basic functions}

\FloatBarrier
\begin{table}[!ht]
\caption{Basic Functions}
\begin{center}
\begin{tabular}{ll}
\hline 
Function & 	Description \\ \hline 
abs(x) & 	absolute value \\
sqrt(x) 	& square root \\
ceiling(x) & 	ceiling(3.475) is 4\\
floor(x) & 	floor(3.475) is 3\\
trunc(x) & 	trunc(5.99) is 5\\
round(x, digits=n) & 	round(3.475, digits=2) is 3.48\\
signif(x, digits=n) & 	signif(3.475, digits=2) is 3.5\\
log(x) & 	logarithm \\
exp(x) & 	$e^x$ \\ \hline 
substr(x, start=n1, stop=n2) 	& Extract or replace substrings in a character vector.\\
&x = "abcdef" , substr(x, 2, 4) is "bcd"\\
grep(pattern, x ) &	Search for pattern in x. \\
sub(pattern, replacement, x) &	Find pattern in x and replace with replacement text. \\
strsplit(x, split) & 	Split the elements of character vector x at split.\\
strsplit("abc", "") & returns 3 element vector "a","b","c"\\
paste(..., sep="")  & 	Concatenate strings\\
toupper(x) &	 Uppercase\\
tolower(x) & 	Lowercase\\
\hline 
\end{tabular} 
\end{center}
\end{table}
\FloatBarrier

\subsection{Language}

\begin{verbatim}
if (condition) statement 
for (i in range) statement
while (condition) statement
fun = function(input) {calculation return(output)} 
fun = function(input) {calculation output} 
\end{verbatim}


\begin{Exercise}[title = Programming]
\begin{verbatim}
Write a program that asks the user to 
type an integer N and compute u(N) defined with :
u(0)=1
u(1)=1
u(n+1)=u(n)+u(n-1)
 \end{verbatim}

\begin{enumerate}
\item Evaluate  $1^2 + 2^2 +3^2 + \ldots 400^2$.
\item Evaluate $1 \times 2+2\times3+3\times4+ ... + 249\times250$
\item Create a function "crra" with two arguments $(c,\theta)$ that returns $\frac{c^{1-\theta}}{1-\theta}$. Add an if condition such that the utility is given by the log when $\theta \in [0.97,1,03] \approx 1$
\item  Create a function "fact" that returns the factorial of a number
\end{enumerate}
\end{Exercise}



\begin{Exercise}[title = Apply Functions]
\begin{enumerate}
\item Using this object, \begin{verbatim}
m = matrix(c(rnorm(20,0,10), rnorm(20,-1,10)), nrow = 20, ncol = 2)
\end{verbatim} Calculate the mean, median, min, max and standard deviation by row and column.
\item Using the dataset iris in the package "datasets", calculate the average \textbf{Sepal.Length} by \textbf{Species}. Evaluate the sum log of \textbf{Sepal.Width} by \textbf{Species}.
\item \begin{verbatim}
y1 = NULL; for (i in 1:100) y1[i]=exp(i)
y2 = exp(1:100)
y3 = sapply(1:100,exp)
\end{verbatim}
\begin{enumerate}
\item Check the outcome of these three operations.
\item Using proc.time() or system.time(), compare the execution time of these three equivalents commands.
\end{enumerate}
\end{enumerate}
\end{Exercise}


\begin{table}
\caption{Apply functions}
\begin{center}
\begin{tabular}{cc}
\hline
Functions    &      Usage\\ \hline
apply        &    Apply Functions Over Array Margins \\
by           &     Apply a Function to a Data Frame Split by Factors \\
eapply       &    Apply a Function Over Values in an Environment \\
lapply       &     Apply a Function over a List or Vector \\
mapply       &     Apply a Function to Multiple List or Vector Arguments \\
rapply       &     Recursively Apply a Function to a List \\
tapply       &     Apply a Function Over a Ragged Array \\
\hline 
\end{tabular} 
\end{center}
\end{table}
\FloatBarrier

\subsection{Statistics}
\FloatBarrier

\begin{table}
\caption{Statistical distributions}
\begin{center}
\begin{tabular}{ll}
name &	description\\ \hline
dname( ) &	density or probability function\\
pname( ) &	cumulative density function\\
qname( ) &	quantile function\\
rname( ) &	random deviates \\ \hline 
\end{tabular} 
\end{center}
\end{table}

\begin{table}
\caption{Statistical Functions}
\begin{center}
\begin{tabular}{ll}
Function & 	Description \\ \hline 
mean(x, trim=0,na.rm=FALSE) &	mean of object x\\
sd(x), var(x) 	& standard deviation, variance of object(x) \\
median(x) &	median \\
quantile(x, probs) & x is the numeric vector and probs is a numeric vector with probabilities \\
range(x) &	range \\
sum(x) &	sum \\
diff(x, lag=1) &	lagged differences, with lag indicating which lag to use \\
min(x) & 	minimum \\
max(x) &	maximum \\ \hline 
\end{tabular} 
\end{center}
\end{table}

\begin{table}
\caption{Statistical distributions}
\begin{center}
\begin{tabular}{ll}
Distribution 	& R name \\ \hline 
Beta 	& beta 	\\
Lognormal &	lnorm \\
Binomial &	binom 	\\
Negative Binomial &	nbinom \\
Cauchy & 	cauchy 	\\
Normal &	 norm \\
Chisquare &	chisq \\ 
Poisson & 	pois \\
Exponential &	exp \\ 
Student t &	t \\
F 	& f 	\\
Uniform &	unif \\
Gamma 	& gamma  \\
Tukey 	& tukey \\
Geometric &	geom \\	
Weibull 	& weib \\
Hypergeometric &	hyper \\	
Wilcoxon &	wilcox \\
Logistic &	logis \\	 \hline 	 
\end{tabular} 
\end{center}
\end{table}

\begin{Exercise}[title = Simulating and Computing]
\begin{enumerate}
\item Simulate a vector x of 10,000 draws from a normal distribution. Use the function summary to provide basic characteristics of x.
\item Create a function \emph{dsummary} that returns, the minimum, the 1st decile, the 1st quartile, the median, the mean, the standard deviation, the 3rd quartile, the 9th decile, and the maximum. 
\item Suppose $X \sim N(2,0.25)$. Evaluate 
 $f(0.5), F(2.5), F^{-1}(0.95)$
\item Repeat if X has t-distribution with 5 degrees of freedom.
\item Suppose $X \sim P(3,1)$, where P is the pareto distribution. Evaluate $f(0.5), F(2.5), F^{-1}(0.95)$
\end{enumerate}
\end{Exercise}



%\begin{Exercise}[title = Tests and indexing]
%\end{Exercise}
%
%\begin{enumerate}
%\end{enumerate}

\begin{Exercise}[title = Moments]
Consider a vector $V = rnorm(100,-2,5)$.
\begin{enumerate}
\item Evaluate n as the length of V.
\item Compute the mean $m = \dfrac{1}{n}\sum_{i=1}^{i=n} V_i$
\item Compute the variance $s^2 = \dfrac{1}{n-1} \sum_i^n (V_i-m)^2$
\item Compute the skewness $\gamma_1 = \dfrac{1}{n} \dfrac{(V_i-m)^3}{s^3}$
\item Compute the kurtosis $k_1 = \dfrac{1}{n} \dfrac{(V_i-m)^4}{s^4} - 3$
\end{enumerate}
\end{Exercise}


\FloatBarrier
\begin{table}
\caption{Matrix operation}
\begin{center}
\begin{tabular}{ll}
Function (Operator) 	& Description \\ \hline 
$A * B$
	& Element wise multiplication \\
$A\%*\%B$
 &	matrix multiplication \\
t(A) &	Transpose \\	
diag(a) &	Create a diagonal matrix with a elements \\
diag(A) &	Return the diagonal of A \\	
Solve(A) &	inverse of A \\	
 \hline 	 
\end{tabular} 
\end{center}
\end{table}

\begin{Exercise}[title = OLS]
\begin{enumerate}
\item Create a matrix \emph{X} of dimension \emph{(1000,10)}. Fill it with draws from a beta distribution with shape1 parameter 2, and shape 2 parameter 1. Make sure that there is no negative. 
\item Create a scalar denoted by $\sigma^2$ and set it to 0.5. Generate a vector $\beta$ of size 10. Fill it with draws from a \emph{Gamma} distribution with parameters 2 and 1. 
\item Create a vector $\epsilon$ of 1000 draws from a normal distribution. 
\item Create $Y=  X\beta + \sqrt{\sigma^2} *\epsilon$
\item Recover $\hat{\beta} = (X'X)^{-1}(X'Y)$
\item Evaluate $\widehat{\epsilon} = \widehat{y}-y$. Plot the histogram (filled in grey) and the kernel density of the distribution of the error term. 
\item Estimate $\sigma^2 = \dfrac{\widehat{\epsilon}'\widehat{\epsilon}}{n-p-1}$, and $\V(\widehat{\beta})= \sigma^2 (X'X)^{-1}$
\item Create \emph{param} that binds $(\beta,\sqrt{V(\widehat{\beta})})$. Using the command \emph{lm}, check these estimates.  
\item Construct a confidence interval for $\beta$. 
\item Redo the exercise by setting $\sigma^2=0.01$. How are your confidence intervals for $\beta$. 
\end{enumerate}
\end{Exercise}

%\subsection{Data manipulation}
%\subsection{Plotting}
%\subsection{Problem Sets}



\end{document}
